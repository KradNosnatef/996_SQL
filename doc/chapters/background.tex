\documentclass[../report.tex]{subfiles}
\begin{document}

\section{技术背景}

在对业界的数据库开发做了一些简单的调研之后,我们得到了如下的一些信息:

\begin{enumerate}
\itemsep -0.3em
\item MySQL 是一般开发人员最喜欢用的数据库之一,也是我们课程推荐的数据库系统
\item 一般的 Web 应用开发人员更多的喜欢使用 Java 进行开发,具有优秀的可移植性
\item 现如今的 Web App 更推崇使用框架,注重敏捷开发
\end{enumerate}

\section {技术方案选择}

在简单了解各个技术方案的优劣之后,我们最终并没有选择一些成熟的框架进行开发。而是使用了一些更加传统的技术方案。

我们的核心技术点为Java Web,这使得我们可以使用安全的 JDBC 接口去保护我们的数据安全。此外,我们使用JSP + Servlet + JavaBean 去规划处理我们的各种业务需求,使得开发可以有序推进。

\section{开发工具}

\begin{enumerate}
\itemsep -0.3em
\item 数据库工具: MySQL Database 8.0+
\item Java 开发工具: IntelliJ IDEA 2019+
\item 前端工具: Brackets、 VS Code、 Vim
\item 后端承载平台: Tomcat 9.0
\item 文档工具: XeLatex
\item 测试工具: IntelliJ Junit
\item 管理工具: Git、 GItg
\end{enumerate}

这里我的选择注重了“经典和流行兼顾,便捷与规范具备”的原则。

Mysql 作为经典的数据库平台工具,是我们最初确定的技术核心。在确定了Java Web作为开发任务之后,选择了较为流行的开发工具:IntelliJ IDEA。我们配置的后端注重了规范性,向这行业的标准靠拢,选择给予Apache的一个简单易用的版本Tomcat进行部署。而前端的编辑,我们不拘一格,各显身手,使用各自习惯的工具进行操作。

用到的编程语言主要为: Java CSS HTML JavaScript

\section{组员分工}

\subsection{前端分工}

页面设计布局:黄业琦

美观优化:张行健、杭晗

\subsection{前后端对接}

前后端对接、Tomcat的配置: 黄业琦

\subsection{后端分工}

数据库链接组件:黄业琦、朱凡

校区、专业、成绩、个人基本信息、教师管理: 黄业琦

复杂查询: 杭晗、 张行健

学生管理、学籍异动管理: 张行健

班级管理: 张欣瑞

课程管理、开课管理: 朱凡

\subsection {其他}

测试: 张欣瑞、张行健、杭晗、朱凡

文档与demo录制: 黄业琦

\end{document}